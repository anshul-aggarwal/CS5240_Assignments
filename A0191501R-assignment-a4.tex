% !TeX spellcheck = en_GB
\documentclass[12pt,a4paper]{article}
\usepackage[margin=1.2in]{geometry}
\usepackage{enumitem}

%opening
\title{Assignment A4}
\author{\textbf{Anshul Aggarwal}\\A0191501R}

\date{CS5240 Theoretical Foundations in Multimedia}

\begin{document}

\maketitle

\section*{Problem Objective}

Given a set of data points, $( x _ { i } , y _ { i } , z _ { i } , v _ { i })$, for $i = 1 , \dots , n$, related by the following equation:
\begin{equation}\label{relationeq}
    v _ { i } = a x _ { i } y _ { i } ^ { b } \exp \left( c z _ { i } + d \right)
\end{equation}

\noindent where $a,b,c$ and $d$ are function parameters. The objective is to arrange the equations into matrix form 
\begin{equation}\label{matrixrep}
\textbf{D} \textbf{u}=\textbf{v}
\end{equation}

\noindent where $\textbf{u}$ contains function parameters, and derive the solutions of $a,b,c$ and $d$ based on linear least squares, expressing the solutions in terms of $\textbf{u}$.

\section*{Solution}

To convert it into matrix form represented by equation \ref{matrixrep}, we need to convert equation \ref{relationeq} into additive form. For simplicity, let us consider $x_i, y_i, v_i \textrm{ and } a$ to be positive numbers. From equation \ref{relationeq}, taking natural logarithm on both sides

\begin{equation}
    \ln v_i = \ln a + \ln x_i + b\ln y_i + cz_i + d
\end{equation}

\noindent Rearranging terms, and combining like terms, we get
%
%\begin{equation}
%    (1) \times \ln a  + (\ln y_i) \times b + (z_i)\times c + (1)\times d + (\ln x_i) \times (1) = \ln v_i
%\end{equation}

\begin{equation}
(1) \times \big(\ln(a) + d\big) + (\ln y_i) \times b + (z_i)\times c + (\ln x_i) \times (1) = \ln v_i
\end{equation}


\noindent It is of note that $\ln (a) \textrm{ and } d$ are constants. Converting into matrix form, we get

% This is incorrect, as the matrix is rank deficient. We need to combine ln a and d

%
%\begin{equation}
%    \left[\begin{array}{ccccc} {1} & {\ln y_i} & {z_i} & {1} & {\ln x_i}\end{array} \right]
%    \left[\begin{array}{c}{\ln a}\\ {b} \\ {c}\\ {d} \\ {1} \end{array}\right] = \left[ \begin{array}{c}\ln v_i \end{array}\right]
%\end{equation}
%


\begin{equation}
\left[\begin{array}{cccc} {1} & {\ln y_i} & {z_i} & {\ln x_i}\end{array} \right]
\left[\begin{array}{c}{\ln(a) + d}\\ {b} \\ {c} \\ {1} \end{array}\right] = \left[ \begin{array}{c}\ln v_i \end{array}\right]
\end{equation}

\noindent Expanding representation using all points, we get
%
%\begin{equation}
%\left[\begin{array}{ccccc} {1} & {\ln y_1} & {z_1} & {1} & {\ln x_1} \\  {1} & {\ln y_2} & {z_2} & {1} & {\ln x_2} \\ & & \vdots & & \\  {1} & {\ln y_n} & {z_n} & {1} & {\ln x_n}\end{array} \right]
%\left[\begin{array}{c}{\ln a}\\ {b} \\ {c}\\ {d} \\ {1} \end{array}\right] = \left[ \begin{array}{c}\ln v_1 \\ \ln v_2 \\ \vdots \\ \ln v_n \end{array}\right]
%\end{equation}


\begin{equation}
\left[\begin{array}{cccc} {1} & {\ln y_1} & {z_1} & {\ln x_1} \\  {1} & {\ln y_2} & {z_2} & {\ln x_2} \\ & \vdots & \vdots & \\  {1} & {\ln y_n} & {z_n} & {\ln x_n}\end{array} \right]
\left[\begin{array}{c}{\ln(a) + d}\\ {b} \\ {c}\\ {1} \end{array}\right] = \left[ \begin{array}{c}\ln v_1 \\ \ln v_2 \\ \vdots \\ \ln v_n \end{array}\right]
\end{equation}

\noindent Which can be written in the form of equation \ref{matrixrep}, where 

%\vspace{1em}
\begin{center}
\fbox{%
    \parbox{0.85\textwidth}{%
%\[ \textbf{D} = \left[\begin{array}{ccccc} {1} & {\ln y_1} & {z_1} & {1} & {\ln x_1} \\  {1} & {\ln y_2} & {z_2} & {1} & {\ln x_2} \\ & & \vdots & & \\  {1} & {\ln y_n} & {z_n} & {1} & {\ln x_n}\end{array} \right], \hspace*{0.5em} \textbf{u} = \left[\begin{array}{c}{\ln a}\\ {b} \\ {c}\\ {d} \\ {1} \end{array}\right], \textrm{ and } \textbf{v} = \left[ \begin{array}{c}\ln v_1 \\ \ln v_2 \\ \vdots \\ \ln v_n \end{array}\right] \]
\[ \textbf{D} = \left[\begin{array}{cccc} {1} & {\ln y_1} & {z_1} & {\ln x_1} \\  {1} & {\ln y_2} & {z_2} & {\ln x_2} \\ & \vdots & \vdots & \\  {1} & {\ln y_n} & {z_n} &  {\ln x_n}\end{array} \right], \hspace*{0.5em} \textbf{u} = \left[\begin{array}{c}{\ln(a) + d}\\ {b} \\ {c} \\ {1} \end{array}\right], \textrm{ and } \textbf{v} = \left[ \begin{array}{c}\ln v_1 \\ \ln v_2 \\ \vdots \\ \ln v_n \end{array}\right] \]
}%
}

\end{center}
\noindent Considering the matrix $\textbf{D}$ is full rank, i.e. rank($\textbf{D}$) = 4, then $\textbf{D}^\top\textbf{D}$ has an inverse. This means
\begin{equation}\label{solutionrep}
\textbf{u}=(\textbf{D}^\top\textbf{D})^{-1} \textbf{D}^\top\textbf{v}
\end{equation}

\subsection*{Note}

\begin{itemize}
    \item $a$ and $d$ cannot be independently solved, as both are part of the constant term in the original equation. Any combination of a and d that satisfies the equality constraint for $\ln (a) + d$ will be a solution.
\end{itemize}



\end{document}
